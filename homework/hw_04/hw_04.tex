% Options for packages loaded elsewhere
\PassOptionsToPackage{unicode}{hyperref}
\PassOptionsToPackage{hyphens}{url}
%
\documentclass[
]{article}
\title{hw\_04}
\author{Samantha Rutledge}
\date{10/26/2021}

\usepackage{amsmath,amssymb}
\usepackage{lmodern}
\usepackage{iftex}
\ifPDFTeX
  \usepackage[T1]{fontenc}
  \usepackage[utf8]{inputenc}
  \usepackage{textcomp} % provide euro and other symbols
\else % if luatex or xetex
  \usepackage{unicode-math}
  \defaultfontfeatures{Scale=MatchLowercase}
  \defaultfontfeatures[\rmfamily]{Ligatures=TeX,Scale=1}
\fi
% Use upquote if available, for straight quotes in verbatim environments
\IfFileExists{upquote.sty}{\usepackage{upquote}}{}
\IfFileExists{microtype.sty}{% use microtype if available
  \usepackage[]{microtype}
  \UseMicrotypeSet[protrusion]{basicmath} % disable protrusion for tt fonts
}{}
\makeatletter
\@ifundefined{KOMAClassName}{% if non-KOMA class
  \IfFileExists{parskip.sty}{%
    \usepackage{parskip}
  }{% else
    \setlength{\parindent}{0pt}
    \setlength{\parskip}{6pt plus 2pt minus 1pt}}
}{% if KOMA class
  \KOMAoptions{parskip=half}}
\makeatother
\usepackage{xcolor}
\IfFileExists{xurl.sty}{\usepackage{xurl}}{} % add URL line breaks if available
\IfFileExists{bookmark.sty}{\usepackage{bookmark}}{\usepackage{hyperref}}
\hypersetup{
  pdftitle={hw\_04},
  pdfauthor={Samantha Rutledge},
  hidelinks,
  pdfcreator={LaTeX via pandoc}}
\urlstyle{same} % disable monospaced font for URLs
\usepackage[margin=1in]{geometry}
\usepackage{color}
\usepackage{fancyvrb}
\newcommand{\VerbBar}{|}
\newcommand{\VERB}{\Verb[commandchars=\\\{\}]}
\DefineVerbatimEnvironment{Highlighting}{Verbatim}{commandchars=\\\{\}}
% Add ',fontsize=\small' for more characters per line
\usepackage{framed}
\definecolor{shadecolor}{RGB}{248,248,248}
\newenvironment{Shaded}{\begin{snugshade}}{\end{snugshade}}
\newcommand{\AlertTok}[1]{\textcolor[rgb]{0.94,0.16,0.16}{#1}}
\newcommand{\AnnotationTok}[1]{\textcolor[rgb]{0.56,0.35,0.01}{\textbf{\textit{#1}}}}
\newcommand{\AttributeTok}[1]{\textcolor[rgb]{0.77,0.63,0.00}{#1}}
\newcommand{\BaseNTok}[1]{\textcolor[rgb]{0.00,0.00,0.81}{#1}}
\newcommand{\BuiltInTok}[1]{#1}
\newcommand{\CharTok}[1]{\textcolor[rgb]{0.31,0.60,0.02}{#1}}
\newcommand{\CommentTok}[1]{\textcolor[rgb]{0.56,0.35,0.01}{\textit{#1}}}
\newcommand{\CommentVarTok}[1]{\textcolor[rgb]{0.56,0.35,0.01}{\textbf{\textit{#1}}}}
\newcommand{\ConstantTok}[1]{\textcolor[rgb]{0.00,0.00,0.00}{#1}}
\newcommand{\ControlFlowTok}[1]{\textcolor[rgb]{0.13,0.29,0.53}{\textbf{#1}}}
\newcommand{\DataTypeTok}[1]{\textcolor[rgb]{0.13,0.29,0.53}{#1}}
\newcommand{\DecValTok}[1]{\textcolor[rgb]{0.00,0.00,0.81}{#1}}
\newcommand{\DocumentationTok}[1]{\textcolor[rgb]{0.56,0.35,0.01}{\textbf{\textit{#1}}}}
\newcommand{\ErrorTok}[1]{\textcolor[rgb]{0.64,0.00,0.00}{\textbf{#1}}}
\newcommand{\ExtensionTok}[1]{#1}
\newcommand{\FloatTok}[1]{\textcolor[rgb]{0.00,0.00,0.81}{#1}}
\newcommand{\FunctionTok}[1]{\textcolor[rgb]{0.00,0.00,0.00}{#1}}
\newcommand{\ImportTok}[1]{#1}
\newcommand{\InformationTok}[1]{\textcolor[rgb]{0.56,0.35,0.01}{\textbf{\textit{#1}}}}
\newcommand{\KeywordTok}[1]{\textcolor[rgb]{0.13,0.29,0.53}{\textbf{#1}}}
\newcommand{\NormalTok}[1]{#1}
\newcommand{\OperatorTok}[1]{\textcolor[rgb]{0.81,0.36,0.00}{\textbf{#1}}}
\newcommand{\OtherTok}[1]{\textcolor[rgb]{0.56,0.35,0.01}{#1}}
\newcommand{\PreprocessorTok}[1]{\textcolor[rgb]{0.56,0.35,0.01}{\textit{#1}}}
\newcommand{\RegionMarkerTok}[1]{#1}
\newcommand{\SpecialCharTok}[1]{\textcolor[rgb]{0.00,0.00,0.00}{#1}}
\newcommand{\SpecialStringTok}[1]{\textcolor[rgb]{0.31,0.60,0.02}{#1}}
\newcommand{\StringTok}[1]{\textcolor[rgb]{0.31,0.60,0.02}{#1}}
\newcommand{\VariableTok}[1]{\textcolor[rgb]{0.00,0.00,0.00}{#1}}
\newcommand{\VerbatimStringTok}[1]{\textcolor[rgb]{0.31,0.60,0.02}{#1}}
\newcommand{\WarningTok}[1]{\textcolor[rgb]{0.56,0.35,0.01}{\textbf{\textit{#1}}}}
\usepackage{graphicx}
\makeatletter
\def\maxwidth{\ifdim\Gin@nat@width>\linewidth\linewidth\else\Gin@nat@width\fi}
\def\maxheight{\ifdim\Gin@nat@height>\textheight\textheight\else\Gin@nat@height\fi}
\makeatother
% Scale images if necessary, so that they will not overflow the page
% margins by default, and it is still possible to overwrite the defaults
% using explicit options in \includegraphics[width, height, ...]{}
\setkeys{Gin}{width=\maxwidth,height=\maxheight,keepaspectratio}
% Set default figure placement to htbp
\makeatletter
\def\fps@figure{htbp}
\makeatother
\setlength{\emergencystretch}{3em} % prevent overfull lines
\providecommand{\tightlist}{%
  \setlength{\itemsep}{0pt}\setlength{\parskip}{0pt}}
\setcounter{secnumdepth}{-\maxdimen} % remove section numbering
\ifLuaTeX
  \usepackage{selnolig}  % disable illegal ligatures
\fi

\begin{document}
\maketitle

\hypertarget{use-the-rvest-r-package-to-scrape-the-schedule-and-materials-table-into-r-from-the-course-webpage-httpsintrodatasci.dlilab.comschedule_materials.-read-the-documentation-of-rvest-so-you-get-a-better-idea-about-the-functions-provided-by-rvest-and-their-usages}{%
\subsection{\texorpdfstring{1. Use the rvest R package to scrape the
schedule and materials table into R from the course webpage
(\url{https://introdatasci.dlilab.com/schedule_materials/}). Read the
documentation of rvest so you get a better idea about the functions
provided by rvest and their
usages}{1. Use the rvest R package to scrape the schedule and materials table into R from the course webpage (https://introdatasci.dlilab.com/schedule\_materials/). Read the documentation of rvest so you get a better idea about the functions provided by rvest and their usages}}\label{use-the-rvest-r-package-to-scrape-the-schedule-and-materials-table-into-r-from-the-course-webpage-httpsintrodatasci.dlilab.comschedule_materials.-read-the-documentation-of-rvest-so-you-get-a-better-idea-about-the-functions-provided-by-rvest-and-their-usages}}

\begin{Shaded}
\begin{Highlighting}[]
\FunctionTok{library}\NormalTok{(tidyverse)}
\end{Highlighting}
\end{Shaded}

\begin{verbatim}
## -- Attaching packages --------------------------------------- tidyverse 1.3.1 --
\end{verbatim}

\begin{verbatim}
## v ggplot2 3.3.5     v purrr   0.3.4
## v tibble  3.1.5     v dplyr   1.0.7
## v tidyr   1.1.4     v stringr 1.4.0
## v readr   2.0.2     v forcats 0.5.1
\end{verbatim}

\begin{verbatim}
## -- Conflicts ------------------------------------------ tidyverse_conflicts() --
## x dplyr::filter() masks stats::filter()
## x dplyr::lag()    masks stats::lag()
\end{verbatim}

\begin{Shaded}
\begin{Highlighting}[]
\FunctionTok{library}\NormalTok{(rvest)}
\end{Highlighting}
\end{Shaded}

\begin{verbatim}
## 
## Attaching package: 'rvest'
\end{verbatim}

\begin{verbatim}
## The following object is masked from 'package:readr':
## 
##     guess_encoding
\end{verbatim}

\begin{Shaded}
\begin{Highlighting}[]
\NormalTok{url\_data }\OtherTok{\textless{}{-}} \StringTok{"https://introdatasci.dlilab.com/schedule\_materials/"}
\NormalTok{url\_data }\SpecialCharTok{\%\textgreater{}\%} 
  \FunctionTok{read\_html}\NormalTok{()}
\end{Highlighting}
\end{Shaded}

\begin{verbatim}
## {html_document}
## <html lang="en">
## [1] <head>\n<meta http-equiv="Content-Type" content="text/html; charset=UTF-8 ...
## [2] <body>\n    <a href="#main">skip to content</a>\n    <noscript>\n  <style ...
\end{verbatim}

\begin{Shaded}
\begin{Highlighting}[]
\NormalTok{css\_selector }\OtherTok{\textless{}{-}} \StringTok{"\#main \textgreater{} table"}
\end{Highlighting}
\end{Shaded}

\begin{Shaded}
\begin{Highlighting}[]
\NormalTok{x}\OtherTok{\textless{}{-}}\NormalTok{ url\_data }\SpecialCharTok{\%\textgreater{}\%} 
  \FunctionTok{read\_html}\NormalTok{() }\SpecialCharTok{\%\textgreater{}\%} 
  \FunctionTok{html\_element}\NormalTok{(}\AttributeTok{css =}\NormalTok{ css\_selector) }\SpecialCharTok{\%\textgreater{}\%} 
  \FunctionTok{html\_table}\NormalTok{()}
\NormalTok{x}
\end{Highlighting}
\end{Shaded}

\begin{verbatim}
## # A tibble: 30 x 5
##    Date   Topic                              Notes     HW    Reading            
##    <chr>  <chr>                              <chr>     <chr> <chr>              
##  1 Aug 24 About the course                   "\U0001f~ "-"   "Leek & Peng 2015" 
##  2 Aug 26 Data science project cycle         "\U0001f~ ""    "Mason and Wiggins~
##  3 Aug 31 Class cancelled because of Hurric~ ""        ""    ""                 
##  4 Sep 2  Class cancelled because of Hurric~ ""        ""    ""                 
##  5 Sep 7  Introduction and install tools     "\U0001f~ ""    "Cooper & Hsing 20~
##  6 Sep 9  Version control with Git           "\U0001f~ ""    "Blischak et al. 2~
##  7 Sep 14 Introduction to GitHub             "\U0001f~ ""    ""                 
##  8 Sep 16 RStudio project and dynamic docum~ "\U0001f~ "01"  "Xie et al, Chapte~
##  9 Sep 21 The file system and basic unix sh~ "\U0001f~ ""    "Allesina & Wilmes~
## 10 Sep 23 R basics: data types, vectors, ma~ "\U0001f~ ""    ""                 
## # ... with 20 more rows
\end{verbatim}

\hypertarget{with-the-extracted-data-frame-create-two-new-columns-based-on-the-date-column-month-and-day.-month-would-be-the-month-abbrevations-from-the-date-column-day-would-be-the-numeric-numbers-from-the-date-column.-although-you-can-use-whatever-approach-to-get-this-done-do-not-enter-them-by-hand-i-suggest-you-try-to-practice-regular-expression-here-sub-or-stringrstr_extract.}{%
\subsection{2. With the extracted data frame, create two new columns
based on the Date column: month and day. month would be the month
abbrevations from the Date column; day would be the numeric numbers from
the Date column. Although you can use whatever approach to get this done
(do not enter them by hand\ldots), I suggest you try to practice regular
expression here (sub() or
stringr::str\_extract()).}\label{with-the-extracted-data-frame-create-two-new-columns-based-on-the-date-column-month-and-day.-month-would-be-the-month-abbrevations-from-the-date-column-day-would-be-the-numeric-numbers-from-the-date-column.-although-you-can-use-whatever-approach-to-get-this-done-do-not-enter-them-by-hand-i-suggest-you-try-to-practice-regular-expression-here-sub-or-stringrstr_extract.}}

\begin{Shaded}
\begin{Highlighting}[]
\FunctionTok{library}\NormalTok{(stringr)}
\NormalTok{x}\SpecialCharTok{$}\NormalTok{day }\OtherTok{\textless{}{-}} \FunctionTok{str\_extract}\NormalTok{(x}\SpecialCharTok{$}\NormalTok{Date, }\StringTok{"}\SpecialCharTok{\textbackslash{}\textbackslash{}}\StringTok{d\{2\}"}\NormalTok{) }
\NormalTok{x}\SpecialCharTok{$}\NormalTok{month }\OtherTok{\textless{}{-}} \FunctionTok{str\_extract}\NormalTok{(x}\SpecialCharTok{$}\NormalTok{Date, }\StringTok{"}\SpecialCharTok{\textbackslash{}\textbackslash{}}\StringTok{D\{3\}"}\NormalTok{)}
\end{Highlighting}
\end{Shaded}

\begin{Shaded}
\begin{Highlighting}[]
\NormalTok{x}\SpecialCharTok{$}\NormalTok{day }\OtherTok{\textless{}{-}} \FunctionTok{as.numeric}\NormalTok{(}\FunctionTok{as.character}\NormalTok{(x}\SpecialCharTok{$}\NormalTok{day))}
\NormalTok{x}
\end{Highlighting}
\end{Shaded}

\begin{verbatim}
## # A tibble: 30 x 7
##    Date   Topic                      Notes    HW    Reading            day month
##    <chr>  <chr>                      <chr>    <chr> <chr>            <dbl> <chr>
##  1 Aug 24 About the course           "\U0001~ "-"   "Leek & Peng 20~    24 Aug  
##  2 Aug 26 Data science project cycle "\U0001~ ""    "Mason and Wigg~    26 Aug  
##  3 Aug 31 Class cancelled because o~ ""       ""    ""                  31 Aug  
##  4 Sep 2  Class cancelled because o~ ""       ""    ""                  NA Sep  
##  5 Sep 7  Introduction and install ~ "\U0001~ ""    "Cooper & Hsing~    NA Sep  
##  6 Sep 9  Version control with Git   "\U0001~ ""    "Blischak et al~    NA Sep  
##  7 Sep 14 Introduction to GitHub     "\U0001~ ""    ""                  14 Sep  
##  8 Sep 16 RStudio project and dynam~ "\U0001~ "01"  "Xie et al, Cha~    16 Sep  
##  9 Sep 21 The file system and basic~ "\U0001~ ""    "Allesina & Wil~    21 Sep  
## 10 Sep 23 R basics: data types, vec~ "\U0001~ ""    ""                  23 Sep  
## # ... with 20 more rows
\end{verbatim}

\hypertarget{with-the-data-frame-generated-from-q2-use-group_by-and-summarise-to-find-out-the-number-of-lectures-for-each-month-order-the-results-by-the-number-of-lectures-high-to-low.}{%
\subsection{3. With the data frame generated from Q2, use group\_by()
and summarise() to find out the number of lectures for each month, order
the results by the number of lectures (high to
low).}\label{with-the-data-frame-generated-from-q2-use-group_by-and-summarise-to-find-out-the-number-of-lectures-for-each-month-order-the-results-by-the-number-of-lectures-high-to-low.}}

\begin{Shaded}
\begin{Highlighting}[]
\NormalTok{y }\OtherTok{\textless{}{-}}\NormalTok{ x }\SpecialCharTok{\%\textgreater{}\%} \FunctionTok{group\_by}\NormalTok{(month) }\SpecialCharTok{\%\textgreater{}\%} \FunctionTok{summarise}\NormalTok{(}\AttributeTok{lecture\_number =} \FunctionTok{n}\NormalTok{()) }\SpecialCharTok{\%\textgreater{}\%} \FunctionTok{arrange}\NormalTok{(}\FunctionTok{desc}\NormalTok{(lecture\_number))}
\NormalTok{y}
\end{Highlighting}
\end{Shaded}

\begin{verbatim}
## # A tibble: 5 x 2
##   month lecture_number
##   <chr>          <int>
## 1 Nov                9
## 2 Sep                9
## 3 Oct                7
## 4 Aug                3
## 5 Dec                2
\end{verbatim}

\hypertarget{for-the-topic-column-split-all-values-into-words-hint-stringrstr_split.-observe-the-values-in-the-topic-column-and-use-regular-expression-to-specify-the-pattern-in-the-stringrstr_split-or-strsplit-function.-once-this-is-done-you-should-get-a-list-of-list-you-can-use-unlist-to-convert-it-into-a-vector-and-name-it-as-words.-use-table-and-sort-to-find-the-top-5-most-frequent-words.}{%
\subsection{4. For the Topic column, split all values into words (hint:
stringr::str\_split()). Observe the values in the Topic column and use
regular expression to specify the pattern in the stringr::str\_split()
or strsplit() function. Once this is done, you should get a list of
list, you can use unlist() to convert it into a vector and name it as
words. Use table() and sort() to find the top 5 most frequent
words.}\label{for-the-topic-column-split-all-values-into-words-hint-stringrstr_split.-observe-the-values-in-the-topic-column-and-use-regular-expression-to-specify-the-pattern-in-the-stringrstr_split-or-strsplit-function.-once-this-is-done-you-should-get-a-list-of-list-you-can-use-unlist-to-convert-it-into-a-vector-and-name-it-as-words.-use-table-and-sort-to-find-the-top-5-most-frequent-words.}}

\begin{Shaded}
\begin{Highlighting}[]
\NormalTok{w }\OtherTok{\textless{}{-}} \FunctionTok{strsplit}\NormalTok{(x}\SpecialCharTok{$}\NormalTok{Topic, }\AttributeTok{split =} \StringTok{" "}\NormalTok{)}
\NormalTok{w}
\end{Highlighting}
\end{Shaded}

\begin{verbatim}
## [[1]]
## [1] "About"  "the"    "course"
## 
## [[2]]
## [1] "Data"    "science" "project" "cycle"  
## 
## [[3]]
## [1] "Class"     "cancelled" "because"   "of"        "Hurricane" "Ida"      
## 
## [[4]]
## [1] "Class"     "cancelled" "because"   "of"        "Hurricane" "Ida"      
## 
## [[5]]
## [1] "Introduction" "and"          "install"      "tools"       
## 
## [[6]]
## [1] "Version" "control" "with"    "Git"    
## 
## [[7]]
## [1] "Introduction" "to"           "GitHub"      
## 
## [[8]]
## [1] "RStudio"   "project"   "and"       "dynamic"   "documents" "with"     
## [7] "R"         "Markdown" 
## 
## [[9]]
## [1] "The"    "file"   "system" "and"    "basic"  "unix"   "shell" 
## 
## [[10]]
## [1] "R"        "basics:"  "data"     "types,"   "vectors," "matrix,"  "data"    
## [8] "frame,"   "etc."    
## 
## [[11]]
## [1] "More"    "R"       "basics:" "lists,"  "dates,"  "etc."   
## 
## [[12]]
## [1] "R"           "programming" "basics:"     "conditional" "statements" 
## 
## [[13]]
## [1] "R"           "programming" "basics:"     "loops,"      "apply"      
## 
## [[14]]
## [1] "Strings"     "and"         "Regular"     "expressions"
## 
## [[15]]
## [1] "API"      "and"      "data"     "scraping"
## 
## [[16]]
## [1] "Data"   "input"  "and"    "output"
## 
## [[17]]
## [1] "Data"         "manipulation" "with"         "R"           
## 
## [[18]]
## [1] "More"         "data"         "manipulation" "with"         "R"           
## 
## [[19]]
## [1] "Data"          "visualization" "with"          "R"            
## 
## [[20]]
## [1] "Exploratory" "data"        "analysis"   
## 
## [[21]]
## [1] "Regression" "methods"   
## 
## [[22]]
## [1] "More"       "on"         "Regression" "methods"   
## 
## [[23]]
## [1] "Write"     "your"      "own"       "functions"
## 
## [[24]]
## [1] "Write"   "your"    "own"     "R"       "package"
## 
## [[25]]
## [1] "Open"       "Science"    "and"        "automating" "things"    
## [6] "with"       "Makefile"  
## 
## [[26]]
## [1] "Ethics"    "in"        "data"      "science"   "(virtual)"
## 
## [[27]]
## [1] "Thanksgiving," "no"            "class"        
## 
## [[28]]
## [1] "Final"        "project"      "presentation"
## 
## [[29]]
## [1] "Final"        "project"      "presentation" "and"          "wrap"        
## [6] "up"          
## 
## [[30]]
## [1] "Final"  "grades" "due"
\end{verbatim}

\begin{Shaded}
\begin{Highlighting}[]
\NormalTok{w1 }\OtherTok{\textless{}{-}} \FunctionTok{unlist}\NormalTok{(w)}
\NormalTok{w1}
\end{Highlighting}
\end{Shaded}

\begin{verbatim}
##   [1] "About"         "the"           "course"        "Data"         
##   [5] "science"       "project"       "cycle"         "Class"        
##   [9] "cancelled"     "because"       "of"            "Hurricane"    
##  [13] "Ida"           "Class"         "cancelled"     "because"      
##  [17] "of"            "Hurricane"     "Ida"           "Introduction" 
##  [21] "and"           "install"       "tools"         "Version"      
##  [25] "control"       "with"          "Git"           "Introduction" 
##  [29] "to"            "GitHub"        "RStudio"       "project"      
##  [33] "and"           "dynamic"       "documents"     "with"         
##  [37] "R"             "Markdown"      "The"           "file"         
##  [41] "system"        "and"           "basic"         "unix"         
##  [45] "shell"         "R"             "basics:"       "data"         
##  [49] "types,"        "vectors,"      "matrix,"       "data"         
##  [53] "frame,"        "etc."          "More"          "R"            
##  [57] "basics:"       "lists,"        "dates,"        "etc."         
##  [61] "R"             "programming"   "basics:"       "conditional"  
##  [65] "statements"    "R"             "programming"   "basics:"      
##  [69] "loops,"        "apply"         "Strings"       "and"          
##  [73] "Regular"       "expressions"   "API"           "and"          
##  [77] "data"          "scraping"      "Data"          "input"        
##  [81] "and"           "output"        "Data"          "manipulation" 
##  [85] "with"          "R"             "More"          "data"         
##  [89] "manipulation"  "with"          "R"             "Data"         
##  [93] "visualization" "with"          "R"             "Exploratory"  
##  [97] "data"          "analysis"      "Regression"    "methods"      
## [101] "More"          "on"            "Regression"    "methods"      
## [105] "Write"         "your"          "own"           "functions"    
## [109] "Write"         "your"          "own"           "R"            
## [113] "package"       "Open"          "Science"       "and"          
## [117] "automating"    "things"        "with"          "Makefile"     
## [121] "Ethics"        "in"            "data"          "science"      
## [125] "(virtual)"     "Thanksgiving," "no"            "class"        
## [129] "Final"         "project"       "presentation"  "Final"        
## [133] "project"       "presentation"  "and"           "wrap"         
## [137] "up"            "Final"         "grades"        "due"
\end{verbatim}

\begin{Shaded}
\begin{Highlighting}[]
\FunctionTok{sort}\NormalTok{(}\FunctionTok{table}\NormalTok{(w1),}\AttributeTok{decreasing=}\ConstantTok{TRUE}\NormalTok{)[}\DecValTok{1}\SpecialCharTok{:}\DecValTok{5}\NormalTok{]}
\end{Highlighting}
\end{Shaded}

\begin{verbatim}
## w1
##       R     and    data    with basics: 
##       9       8       6       6       4
\end{verbatim}

\end{document}
